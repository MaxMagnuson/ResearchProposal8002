\documentclass[]{article}

\usepackage{algorithm}
\usepackage[noend]{algpseudocode}
\usepackage{cite}
\usepackage{graphicx}
\usepackage{xcolor}

% Use commented out command when done
%\newcommand{\comment}[1]{}
\newcommand{\comment}[1]
{\par {\bfseries \color{green} #1 \par}}

%opening

\title{Narrative Generation Research Proposal}
\author{Max Magnuson}

\begin{document}
\maketitle

\section{Motivation}
Narrative generation as the name would imply is the process of algorithmically generating new stories. Narrative generation has many intriguing impacts in the area of interactive media. Endless amounts of reading material could be generated for entertainment or learning purposes which are catered to the user's interests. In video games, it can create a rich and dynamic environment for the player. Since the plot changes and reacts to the actions of the player, it makes the player really feel like their actions matter. This lends itself to a deeper, more immersive experience.

\section{Vision}
Narrative generation requires many different areas of computer science to be successful. It requires elements of artificial intelligence, so that each character can reason and choose actions that are believable. It requires natural language processing to convey the plot and events to the user in an interesting way. It would be boring if the system simply stated every action that happens. When it comes to video games, it might require some form of computer graphics to display the narrative instead of only using text. 

With so many fields involved, I believe it to be critical that each piece of the system be made sufficiently modular. On top of that, narrative generation needs a flexible platform for all of the different modules. When designed in this manner, we can take advantage of discoveries in each of the respective fields and incorporate them into the appropriate module. 

\bibliography{mybib}
\bibliographystyle{plain}
\end{document}